\documentclass[a4paper]{article}

% upto two sides of A4

\usepackage{color}
\usepackage{eepic}
\usepackage{times} \usepackage{mathptm}  
\usepackage{amssymb}
\usepackage{latexsym}
\usepackage{url}
\RequirePackage{vmargin}
\setmarginsrb{1.9cm}{2cm}{1.9cm}{2cm}{0pt}{0pt}{0pt}{0pt} %jg 2cm all round
  %ctm shaves another mil left and right
\setlength{\columnsep}{1cm}

%jg redefined title layout
\makeatletter
\def\maketitle{%
\begin{flushleft}
\LARGE\bfseries \@title \bigskip \\
\Large\mdseries \@author \medskip
\end{flushleft}%
}
\makeatother

\setlength{\marginparwidth}{1cm}
\setlength{\marginparsep}{3mm}
\def\marginnote#1{\marginpar{\scriptsize\itshape\raggedright#1}}

%%%\renewcommand{\baselinestretch}{0.9641}\small\normalsize
% \renewcommand{\baselinestretch}{0.972373}\small\normalsize %jg ick!
\makeatletter
%%%\setlength{\parskip}{0\p@ \@plus 0.98\p@}
% \setlength{\parskip}{0.75\parskip} %jg prefer a parindent
\renewcommand{\section}{\@startsection
 {section}%
 {1}%
 {0mm}%
 {-\baselineskip}%
 {0.5\baselineskip}%
 {\Large\bfseries}%
 }%
\renewcommand{\subsection}{\@startsection
 {subsection}%
 {2}%
 {0mm}%
 {-0.75\baselineskip}%
 {0.375\baselineskip}%0.01
 {\large\bfseries}%
 }%
\makeatother

\pagestyle{empty}

% \parskip 0.05in \parindent 0in %jg prefer a parindent
\newcommand{\sparadrap}[1]{\par\mbox{\textbf{#1}$\;\;$}}
\newcommand{\squashit}{\itemsep=0pt
  \topsep=0pt \parskip=0pt \parsep=0pt}
\let\ITEMIZE\itemize \def\itemize{\ITEMIZE\squashit}

\newcommand{\remph}{\emph}
\newcommand{\demph}{\textbf}

\newcommand{\bsize}[2]{{#1\textbf{#2}\par}}
\newcommand{\bL}{\vspace{0.1in}\noindent\bsize{\Large}}
\newcommand{\bl}{\bsize{\large}}
\newenvironment{titemize}{
  \begin{list}{\ensuremath{\bullet}}{
  \setlength{\topsep}{1pt}
  \setlength{\parsep}{0pt}
  %\setlength{\parskip}{0pt}
  \setlength{\itemsep}{1pt}
  \setlength{\partopsep}{0pt}}
}
{
  \end{list}
}

\usepackage{lscape}
\usepackage{pstricks}
\usepackage{pst-grad}
%%\usepackage{pst-xkey}
\usepackage{multido}

\makeatletter
\makeatother

\newcommand{\surl}[1]{\small{\url{#1}}}
\newcommand{\parametris}{parametriz}




\title{Quantum Types for Quantum Programs:\\
 \Large Towards a Framework for Certified Quantum Information Processing
}
\author{(Pathways to impact) --- Thorsten Altenkirch and Alexander S. Green}
\date{}

\begin{document}

\maketitle

Quantum computing is an emerging technology which has a potentially
immeasurable impact on our computational resources and hence on the
economy as a whole. Quantum computers may lead to an unprecedented
speedup to solve algorithmic problems while quantum protocols may
impact drastically on the efficiency, safety and secrecy of
information protocols. In general, Quantum computation is of interest
to many types of organisation, from financial institutions who rely on
secure communications, and chemical and drug manufactures who need to
simulate complex molecules at the quantum mechanical level, to
governments and intelligence agencies, such as IARPA who recently ran
the QCS project \cite{QCS}. In particular, Quantum Programming
Languages allow for the formalisation of abstract quantum algorithms,
as well as calculating the size of quantum resources that would be
required to run these algorithms (see \cite{quipper-ns}, a New
Scientist article giving an overview of how Quipper has been implemented
for these tasks).

On the other hand there is the emerging area of interactive formal
development based on Type Theory: here we are concerned with the
development of novel high level programming languages which can
provide machine checkable certificates that a program does what it
says. Impressive achievements include the verification of a C-compiler
\cite{compcert-back},  of an operating systems kernel
\cite{seL4_CACM_10} and the proof of a famously hard mathematical
problem (the 4 colour theorem)
\cite{gonthier:four-colour-paper}. Companies are also using these
techniques based on Dependent Type Theory to produce embedded and
safety-critical software for customers, such as Functor AB
(\texttt{http://www.functor.se/}) who have a partnership with the CCFE
and the JET fusion reactor, where the software is used in real-time
systems that deal with the stabilization of the reactor's plasma.

We believe that there is an exciting potential impact arising from the
interaction of these two areas. We give some examples:

\subsection*{Verifiability of claims in Quantum Information Processing}
\label{sec:do-you-realy}

Our project addresses the reliability and
trustworthiness of quantum computation - this is even more relevant in
a novel area where the public, business people, journalists and
politicians cannot rely on previous experiences with a certain
technology. 

As an example, one may consider a number of highly
disputed claims in this area such as D-Wave's adiabatic system
\cite{dwave}. Hence we estimate that our research, in conjunction with
Kashefi's work on blind quantum computation \cite{blind,blind2}, 
could contribute to the ability to certify promises made by companies
providing quantum computing resources.


\subsection*{Secure quantum communication}
\label{sec:secure-quant-comm}

Another area is secure communication, an area which becomes
increasingly more important now that many essential transactions are
handled via digital networks. The emerging technology of securing
communication channels using quantum resources hence has a clear
economic impact \cite{qci}. Indeed, there are many
new protocols being developed but only very few are verified. We need
to avoid the expensive mistakes of the past (i.e. the proliferation of
computer viruses and phishing emails, identity theft and hackers
breaking into bank networks and even stealing pins from ATMs).

Developing a certifiable quantum technology for safe communication
would be even more essential once quantum computers become available
since using them, hackers could break most of the classical
cryptographic protocols that we currently rely upon.

\subsection*{Quantum software engineering}
\label{sec:quant-softw-engin}

Our framework provides the opportunity to perform experiments with
quantum computing resources which are currently hard or expensive to
construct. It also provides an alternative to quantum simulators, which
suffer from the problem that quantum systems are computationally hard
to simulate on a classical computer. We also hope that our framework
will contribute to more structured and effective ways of constructing
quantum software. 

It is already the case that formalizing an algorithm within a
programming language can help find errors in the algorithm
specification, and this extends  naturallyto the quantum realm. For
example, the programming language Quipper was used to implement the
Ground State Estimation algorithm \cite{GSE}, and this process lead to
the discovery of a couple of errors in the circuits presented in the
paper. Our framework would go further, as using a stronger type
system to derive formal proofs of software correctness allows even
more scope for detecting (and fixing) errors in an algorithm's
specification.

\subsection*{Teaching Quantum Information Processing}
\label{sec:teaching-qip}

We have already exploited the Haskell QIO monad to teach quantum
computing at Nottingham. This worked very well in conjunction with a
new, excellent textbook in the area \cite{Mermin}. We found that using
the QIO software clearly enhanced the student's understanding and gave
them the opportunity to develop skills instead of merely textbook
knowledge. The QIO framework we would be developing in
this project would provide further opportunities to educate students
and the interested public in quantum information processing. 

\subsection*{Quantum Information Processing: a killer app for Type Theory ?}
\label{sec:qip-killer-app}

It is frequently the case that new methods (here Type Theory) find it
difficult to replace existing technology (here conventional software
engineering) in conventional areas. This is due to the fact that
initially the new methods are not as effective as the already
established ones, even though they may offer benefits in the long
run. Hence, new methods may have a better chance to be adopted in a
novel application area and can spread from there to other, conventional
engineering problems. In this sense, quantum computing may provide the
killer application for type-theoretic software engineering.

\bibliographystyle{abbrv}
\bibliography{proposal} 

\end{document}
