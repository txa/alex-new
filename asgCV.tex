\documentclass[line]{res}
\usepackage{graphicx,wrapfig}
\usepackage{hyperref}
\hypersetup{colorlinks,urlcolor=black}

\newenvironment{list1}{
  \begin{list}{\ding{113}}{%
      \setlength{\itemsep}{0in}
      \setlength{\parsep}{0in} \setlength{\parskip}{0in}
      \setlength{\topsep}{0in} \setlength{\partopsep}{0in}
      \setlength{\leftmargin}{0.17in}}}{\end{list}}
\newenvironment{list2}{
  \begin{list}{$\bullet$}{%
      \setlength{\itemsep}{0in}
      \setlength{\parsep}{0.045in} \setlength{\parskip}{0in}
      \setlength{\topsep}{0in} \setlength{\partopsep}{0in}
      \setlength{\leftmargin}{0.2in}}}{\end{list}}

\begin{document}

\name{Dr. Alexander S. Green}

\begin{resume}

\section{\sc Contact Details}
\begin{ncolumn}{2}
39 Weston Road, & {\it Phone:}  +44 (0)1275 392144 
   \hspace{3mm} {\it Mobile:}  +44 (0)7743 855930 \\
Long Ashton, Bristol, BS41 9AA & 
   {\it Email:}~\texttt{alexander.s.green@gmail.com} \\
\end{ncolumn}

\vspace{-4mm}
\section{\sc Employment History}
\begin{format}
\title{l}\employer{r}\\
\dates{l}\location{r}\\
\body\\
\end{format}

\employer{Department of Mathematics and Statistics}
\location{Dalhousie University, Halifax, NS, Canada}
\title{Postdoctoral Fellow}
\dates{November 2011 - August 2013}
\begin{position}
\vspace{-4mm}
\begin{list2}
\item I have worked on developing Quipper, a scalable quantum
  programming language, embedded in the functional language Haskell.
\url{http://www.mathstat.dal.ca/~selinger/quipper/}
\item We have used Quipper to implement seven complex quantum
  algorithms from the literature.
\item The project was funded through IARPA's Quantum Computer 
 Science Program (QCS).
\item I have attended QCS program workshops, and presented 
  our work at Logic and interactions 2012, at the CIRM, in 
  Marseille, France.
\item We have had two papers published relating to Quipper, one of
  which I presented at Reversible Computation 2013 in
  Victoria, British Columbia, Canada.
\end{list2}
\end{position}

\employer{Nexor Ltd.}\location{Nottingham, UK}
\title{Graduate Software Engineer}
\dates{November 2010 - November 2011}
\begin{position}
\vspace{-4mm}
\begin{list2}
\item Nexor connects, transforms and protects sensitive information to
  ensure trusted access and secure interoperability for defence and
  government agencies. 
\item My role mainly involved programming using the language C++,  and
  as such, I attended a C++ training course. This included usage of
  some of the up and coming features of C++0x.
\item I also used SQL and wrote code for use in the SAS
  Business Analytics suite of software, as well as shell scripting for
  bash.
\item I was encouraged with my participation in the Intellect
  Young Professionals Network, and have been in schools as a Cyber
  Champion for the UK Safer Internet Centre's Safer Internet Day.
\end{list2}
\end{position}

\employer{School of Computer Science} 
\location{University of Nottingham, UK}
\title{Lecturer for G53NSC and G54NSC}
\dates{January 2010 - July 2010}
\begin{position}
\vspace{-4mm}
\begin{list2}
\item These were 3rd year undergraduate, and masters level 
  modules on Non-Standard Computation, introducing Quantum
  Computation to students with a Computer Science background.
\item I prepared all the course material, including lecture slides and
  the portfolio project material, which made use of Haskell, and the 
  Quantum IO Monad.
\item I graded research papers written by the students on topics
  in the area of Quantum Computation.
\end{list2}
\end{position}

\employer{working for Dr Li Bai} \location{University of Nottingham, UK}
\title{Summer Research Position}
\dates{June 2005 - August 2005}
\begin{position}
\vspace{-4mm}
\begin{list2}
\item I researched the area of GPS navigation to aid in an Augmented
  Reality project.
\item The work included using relatively large geographic data sets,
  as well as looking into cloud computation, and distributed parallel
  systems.
\end{list2}
\end{position}

\vspace{-4mm}
\section{\sc Skills and Experience}
\begin{list2}
\item Extensive programming experience with the functional programming
  language Haskell.
\item Industry experience with the programming language C++.
\item Presenting my own work to both specialist and general computer
  science audiences.
\item Lecturing to computer science undergraduate and masters students
  on quantum computation. 
\item Leading labs and tutorials for undergraduate courses on computer
  systems architecture, mathematics for computer scientists, and
  functional programming in Haskell.
\item Programming with the dependently typed programming language
  Agda, and other programming languages including Java, Scala, OCaml, 
  F\#, and PHP.
\item I have passed the online Coursera modules: Introduction to
  Finance, and Functional Programming Principles in Scala.
\item Document preparation using office suites and LaTeX.
\end{list2}

\vspace{-4mm}
\section{\sc Education}
\begin{format}
\title{l}\employer{r}\\
\dates{l}\location{r}\\
\body\\
\end{format}

\employer{University of Nottingham, UK} \location{}
\title{\bf{PhD in Computer Science}}
\dates{September 2005 - July 2010}
\begin{position}
Thesis title: 
{\em Towards a formally verified functional quantum programming
  language}\\
Thesis on-line at
 \url{http://www.cs.nott.ac.uk/~asg/pdfs/thesis.pdf}\\
\vspace{-4mm}

\begin{list2}
\item My PhD studies focused on developing the Quantum IO Monad, a
  monadic interface to quantum computation, under the supervision of
  Thorsten Altenkirch. 
\url{http://hackage.haskell.org/package/QIO}
\item I also studied advanced functional programming techniques, and
  type theory, including dependent types. As well as attending courses
  at the Midlands Graduate School.
\item I have published a peer-reviewed paper as well as a chapter in
  the book \emph{Semantic Techniques in Quantum Computation}. 
\item I have presented my work at international conferences, including
  BCTCS 2006 and 2007, QPL 2006, QNET workshops 2007; 2008; and 2009,
  QICS workshop 2008, and TFP 2008.
\item I received funding from the EPSRC Network on Semantics of
  Quantum Computation, and the Foundational Structures in Quantum
  Information and Computation STREP grant. 
\end{list2}
\end{position}

\vspace{-2mm}
\employer{University of Nottingham, UK} \location{}
\title{\bf{BSc (Hons) Upper Second Class in Computer Science}}
\dates{September 2001 - July 2005}
\begin{position}
\vspace{-4mm}
\begin{list2}
\item I obtained high marks in modules involving mathematics,
  functional programming, and formal verification. 
\end{list2}
\end{position}

\vspace{-4mm}
\section{\sc Publications}
\begin{list2}
\item An Introduction to Quantum Programming in Quipper
\emph{Alexander S. Green, Peter LeFanu Lumsdaine, Neil J. Ross, 
 Peter Selinger and Beno\^{i}t Valiron}
To appear in the proceedings of the 5th Conference on Reversible Computation, July 2013.
\url{http://arxiv.org/abs/1304.5485}

\item Quipper: A Scalable Quantum Programming Language
\emph{Alexander S. Green, Peter LeFanu Lumsdaine, Neil J. Ross, 
 Peter Selinger and Beno\^{i}t Valiron}
To appear in the proceedings of the 34th annual ACM SIGPLAN conference on Programming Language Design and Implementation, June 2013.
\url{http://arxiv.org/abs/1304.3390}

\item The Quantum IO Monad
\emph{Thorsten Altenkirch and Alexander S. Green}
Semantic Techniques in Quantum Computation 
\emph{edited by Simon Gay and Ian Mackie}
Cambridge University Press, 2010
\emph{ISBN-13: 9780521513746}
\url{http://www.cs.nott.ac.uk/~asg/pdfs/chapter.pdf}

\item From Reversible to Irreversible Computations 
\emph{Alexander S. Green and Thorsten Altenkirch}
Electronic Notes in Theoretical Computer Science, Volume 210, July
2008
\emph{Proceedings of the 4th International Workshop on Quantum
  Programming Languages (QPL 2006)} 
\url{http://www.cs.nott.ac.uk/~asg/pdfs/qpl06.pdf}
\end{list2}

\vspace{-4mm}
\section{\sc Personal Information}
I have always had a keen interest in the subjects of Computer Science
and Mathematics, which helped guide the direction of my studies. For
my PhD I studied a lot of the Physics involved in quantum 
computation, and used my skills as a functional programmer to develop
the Quantum IO Monad. I enjoyed a year working as a
programmer with Nexor, before making the decision to move to Canada
for a Postdoctoral fellowship at Dalhousie University. This has
allowed me to once again use my skills in functional programming to
help develop Quipper. I have also been able to participate in two of my
other main interests: Rugby Union, and Skiing. Any other spare time I
have is often used to program, including toy Android Apps, 
editing GPS logs, and learning new languages.

\vspace{-4mm}
\section{\sc References}
References are available upon request.

\end{resume}
\end{document}
