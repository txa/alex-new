\documentclass[a4paper]{article}

\usepackage{color}
\usepackage{pstcol}
\usepackage{pst-text}

% braket.sty          Macros for Dirac bra-ket <|> notation and sets {|}
% Donald Arseneau     asnd@triumf.ca     Last modified 27-Jan-2003.
% This is free, unencumbered, unsupported software.
%
% Commands defined are:
% \bra{ }   \ket{ }   \braket{ }   \set{ }    (small versions)
% \Bra{ }   \Ket{ }   \Braket{ }   \Set{ }    (expanding versions)
% 
% The "small versions" use fixed-size brackets independent of their
% contents, whereas the "expanding versions" make the brackets and 
% vertical lines expand to envelop their contents (internally using 
% the \left and \right commands).  You should use the vertical bar
% character "|" to input any extra vertical lines.  In \Braket these
% vertical lines will expand to match the arguments, and in \Set the
% first vertical will expand this way.  E.g.,
%   \Braket{ \phi | \frac{\partial^2}{\partial t^2} | \psi }
%   \Set{ x\in\mathbf{R} | 0<{|x|}<5 }
%
% NOT defined is "\ketbra" (for projection operators) because I prefer
% \ket{ } \bra{ }.
%
% Because each definition is so small, it makes no sense to have a 
% complicated generic version for many bracket styles.  Instead, 
% you can just copy the definitions and change \langle or \rangle,
% < and > to what you like.
%
\def\bra#1{\mathinner{\langle{#1}|}}
\def\ket#1{\mathinner{|{#1}\rangle}}
\def\braket#1{\mathinner{\langle{#1}\rangle}}
\def\Bra#1{\left<#1\right|}
\def\Ket#1{\left|#1\right>}
{\catcode`\|=\active 
  \gdef\Braket#1{\begingroup \mathcode`\|32768\let|\BraVert\left<{#1}\right>\endgroup}
}
\def\BraVert{\egroup\,\mid@vertical\,\bgroup}
% The \mid@vertical is \vrule with ordinary TeX but \middle| in eTeX.
% We always avoid a \mathchoice in making the inner vertical lines.  
% Note that \right>, prints the same as \right\rangle but is faster.  
%
% \def\ketbra#1#2{\ket{#1}\bra{#2}}
% \def\Ketbra#1#2{\left|{#1}\vphantom{#2}\right>\left<{#2}\vphantom{#1}\right|}

% \Set{...|...} Only the first | is treated specially.
{\catcode`\|=\active
  \gdef\set#1{\mathinner{\lbrace\,{\mathcode`\|"8000\let|\midvert #1}\,\rbrace}}
  \gdef\Set#1{\left\{\:{\mathcode`\|"8000\let|\SetVert #1}\:\right\}}}
\def\midvert{\egroup\mid\bgroup}
\def\SetVert{\egroup\;\mid@vertical\;\bgroup}

% If the user is using e-TeX with its \middle primitive, use that for
% verticals instead of \vrule.
%
%\begingroup
% \edef\@tempa{\meaning\middle}
% \edef\@tempb{\string\middle}
%\expandafter \endgroup \ifx\@tempa\@tempb
% \def\mid@vertical{\middle|}
%\else
% \let\mid@vertical\vrule
%\fi

\usepackage{eepic}
\usepackage{times} \usepackage{mathptm}  
\usepackage{amssymb}
\usepackage{latexsym}
\usepackage{url}
\RequirePackage{vmargin}
\setmarginsrb{1.9cm}{2cm}{1.9cm}{2cm}{0pt}{0pt}{0pt}{0pt} %jg 2cm all round
  %ctm shaves another mil left and right
\setlength{\columnsep}{1cm}

%jg redefined title layout
\makeatletter
\def\maketitle{%
\begin{flushleft}
\LARGE\bfseries \@title \bigskip \\
\Large\mdseries \@author \medskip
\end{flushleft}%
}
\makeatother

\setlength{\marginparwidth}{1cm}
\setlength{\marginparsep}{3mm}
\def\marginnote#1{\marginpar{\scriptsize\itshape\raggedright#1}}

%%%\renewcommand{\baselinestretch}{0.9641}\small\normalsize
% \renewcommand{\baselinestretch}{0.972373}\small\normalsize %jg ick!
\makeatletter
%%%\setlength{\parskip}{0\p@ \@plus 0.98\p@}
% \setlength{\parskip}{0.75\parskip} %jg prefer a parindent
\renewcommand{\section}{\@startsection
 {section}%
 {1}%
 {0mm}%
 {-\baselineskip}%
 {0.5\baselineskip}%
 {\Large\bfseries}%
 }%
\renewcommand{\subsection}{\@startsection
 {subsection}%
 {2}%
 {0mm}%
 {-0.75\baselineskip}%
 {0.375\baselineskip}%0.01
 {\large\bfseries}%
 }%
\makeatother

\pagestyle{empty}

% \parskip 0.05in \parindent 0in %jg prefer a parindent
\newcommand{\sparadrap}[1]{\par\mbox{\textbf{#1}$\;\;$}}
\newcommand{\squashit}{\itemsep=0pt
  \topsep=0pt \parskip=0pt \parsep=0pt}
\let\ITEMIZE\itemize \def\itemize{\ITEMIZE\squashit}

\newcommand{\remph}{\emph}
\newcommand{\demph}{\textbf}

\newcommand{\bsize}[2]{{#1\textbf{#2}\par}}
\newcommand{\bL}{\vspace{0.1in}\noindent\bsize{\Large}}
\newcommand{\bl}{\bsize{\large}}
\newenvironment{titemize}{
  \begin{list}{\ensuremath{\bullet}}{
  \setlength{\topsep}{1pt}
  \setlength{\parsep}{0pt}
  %\setlength{\parskip}{0pt}
  \setlength{\itemsep}{1pt}
  \setlength{\partopsep}{0pt}}
}
{
  \end{list}
}

\usepackage{lscape}
\usepackage{pstricks}
\usepackage{pst-grad}
%%\usepackage{pst-xkey}
\usepackage{multido}

\makeatletter
\makeatother

\newcommand{\surl}[1]{\small{\url{#1}}}
\newcommand{\parametris}{parametriz}





\title{Quantum Information Processing Meets Type Theory:\\
 \Large A Proposal for a Framework for Certified Quantum Information Processing\\
\LARGE (Case for Support)}
\author{Thorsten Altenkirch and Alexander S. Green}
\date{}

\begin{document}

\twocolumn[\maketitle
\section*{Summary}
Quantum Information Processing is an emerging technology 
\vspace{2em}]

\section*{Graphics}

\subsection*{QIO framework}

This graphic is to demonstrate the relationship between
the various parts that make up the QIO framework\\
\begin{pspicture}(-1,-1)(7,10)

\rput(1.6,6.1){\psline[linewidth=2pt]{<->}(0.7,1.6)}
\pstextpath{
\rput(1.3,6.3){\psline[linewidth=1pt,linecolor=white]{-}(0.84,1.92)}
}{
Reasoning
}

\rput(3.7,7.7){\psline[linewidth=2pt]{<->}(0.7,-1.6)}
\pstextpath{
\rput(4,8){\psline[linewidth=1pt,linecolor=white]{-}(0.84,-1.92)}
}{
Programming
}

\rput(1.6,2.9){\psline[linewidth=2pt]{<->}(0.7,-1.6)}
\pstextpath{
\rput(1.3,2.8){\psline[linewidth=1pt,linecolor=white]{-}(0.84,-1.92)}
}{
Simulation
}

\rput(3.7,1.3){\psline[linewidth=2pt]{<->}(0.7,1.6)}
\pstextpath{
\rput(4,1.1){\psline[linewidth=1pt,linecolor=white]{-}(0.84,1.92)}
}{
Compilation
}

\pscircle[linecolor=black,fillcolor=cyan,fillstyle=solid](3,8.5){1}
\rput(3,8.5){\psscaleboxto(1.3,0){User}}

\psframe[linecolor=black,fillcolor=cyan,fillstyle=solid](1,3)(5,6)
\rput(3,4.9){\psscaleboxto(2,0){QIO}}
\rput(3,4.1){\psscaleboxto(3,0){framework}}

\pscircle[linecolor=black,fillcolor=cyan,fillstyle=solid](3,0.5){1}
\rput(3,0.8){\psscaleboxto(1.2,0){Hardware}}
\rput(3,0.5){\psscaleboxto(0.2,0){or}}
\rput(3,0.2){\psscaleboxto(1.7,0){Physical Model}}


\end{pspicture}


\subsection*{QIO structure}

This graphics demonstrates the structures used in the Quantum IO Monad.

\begin{pspicture}(-1,-1)(7,7)

\psframe[framearc=0.3](0.1,0)(2.6,2.5)
\rput(1.35,2.0){\psscaleboxto(2.2,0){U Monoid}}
\rput(1.35,1.2){\psscaleboxto(2,0){reversible}}
\rput(1.35,0.4){\psscaleboxto(2,0){deterministic}}

\rput(1.35,3.3){\psline[linewidth=2pt]{->}(0,-0.8)}
\pstextpath{
\rput(1.75,3.4){\psline[linewidth=1pt,linearc=0.8,linecolor=white]{->}(0,-0.9)}
}{
apply
}

\psframe[framearc=0.3](0,3.3)(2.7,6)
\rput(1.35,5.55){\psscaleboxto(2.4,0){QIO Monad}}
\rput(1.35,4.65){\psscaleboxto(2,0){irreversible}}
\rput(1.35,3.75){\psscaleboxto(2,0){probabilistic}}

\psline[linewidth=2pt,linearc=0.8]{->}(2.7,4.3)(3.5,4.6)(4.5,4.25)
\pstextpath{
\psline[linewidth=1pt,linearc=0.8,linecolor=white]{->}(2.7,4.6)(3.5,4.9)(4.5,4.55)
}{
Initialisation
}

\psline[linewidth=2pt,linearc=0.8]{<-}(2.1,3.3)(3.1,2.9)(4,3.6)
\pstextpath{
\psline[linewidth=1pt,linearc=0.8,linecolor=white]{-}(2.1,3.0)(3.1,2.7)(4,3.3)
}{
Measurement
}


\psline[linewidth=2pt]{<->}(2.6,1.6)(4,2.5)
\pstextpath{
\psline[linewidth=1pt,linearc=0.8,linecolor=white]{-}(2.7,1.3)(4.1,2.2)
}{
Modifies
}


\psframe[shadow=true,shadowsize=8pt](4,1.75)(6,4.25)
\rput(5,3.5){\psscaleboxto(1.2,0){Qubit}}
\rput(5,2.5){\psscaleboxto(1.2,0){Store}}

\end{pspicture}

\end{document}
