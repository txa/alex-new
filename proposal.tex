\documentclass[a4paper]{article}

\usepackage{pstcol}
\usepackage{pst-text}
\usepackage{color}

\usepackage{multibib}
\newcites{track}{Publications}
\newcites{main}{References}

% \newcommand{\citetrack}[1]{\cite{#1}}
% \newcommand{\citemain}[1]{\cite{#1}}


%    Q-circuit version 1.01
%    Copyright (C) 2004  Steve Flammia & Bryan Eastin

%    This program is free software; you can redistribute it and/or modify
%    it under the terms of the GNU General Public License as published by
%    the Free Software Foundation; either version 2 of the License, or
%    (at your option) any later version.
%
%    This program is distributed in the hope that it will be useful,
%    but WITHOUT ANY WARRANTY; without even the implied warranty of
%    MERCHANTABILITY or FITNESS FOR A PARTICULAR PURPOSE.  See the
%    GNU General Public License for more details.
%
%    You should have received a copy of the GNU General Public License
%    along with this program; if not, write to the Free Software
%    Foundation, Inc., 59 Temple Place, Suite 330, Boston, MA  02111-1307  USA

%   Changed by jjg@cs.nott.ac.uk to import the bra-ket style, rather than define its own

\usepackage[matrix,frame,arrow]{xy}
\usepackage{amsmath}
% braket.sty          Macros for Dirac bra-ket <|> notation and sets {|}
% Donald Arseneau     asnd@triumf.ca     Last modified 27-Jan-2003.
% This is free, unencumbered, unsupported software.
%
% Commands defined are:
% \bra{ }   \ket{ }   \braket{ }   \set{ }    (small versions)
% \Bra{ }   \Ket{ }   \Braket{ }   \Set{ }    (expanding versions)
% 
% The "small versions" use fixed-size brackets independent of their
% contents, whereas the "expanding versions" make the brackets and 
% vertical lines expand to envelop their contents (internally using 
% the \left and \right commands).  You should use the vertical bar
% character "|" to input any extra vertical lines.  In \Braket these
% vertical lines will expand to match the arguments, and in \Set the
% first vertical will expand this way.  E.g.,
%   \Braket{ \phi | \frac{\partial^2}{\partial t^2} | \psi }
%   \Set{ x\in\mathbf{R} | 0<{|x|}<5 }
%
% NOT defined is "\ketbra" (for projection operators) because I prefer
% \ket{ } \bra{ }.
%
% Because each definition is so small, it makes no sense to have a 
% complicated generic version for many bracket styles.  Instead, 
% you can just copy the definitions and change \langle or \rangle,
% < and > to what you like.
%
\def\bra#1{\mathinner{\langle{#1}|}}
\def\ket#1{\mathinner{|{#1}\rangle}}
\def\braket#1{\mathinner{\langle{#1}\rangle}}
\def\Bra#1{\left<#1\right|}
\def\Ket#1{\left|#1\right>}
{\catcode`\|=\active 
  \gdef\Braket#1{\begingroup \mathcode`\|32768\let|\BraVert\left<{#1}\right>\endgroup}
}
\def\BraVert{\egroup\,\mid@vertical\,\bgroup}
% The \mid@vertical is \vrule with ordinary TeX but \middle| in eTeX.
% We always avoid a \mathchoice in making the inner vertical lines.  
% Note that \right>, prints the same as \right\rangle but is faster.  
%
% \def\ketbra#1#2{\ket{#1}\bra{#2}}
% \def\Ketbra#1#2{\left|{#1}\vphantom{#2}\right>\left<{#2}\vphantom{#1}\right|}

% \Set{...|...} Only the first | is treated specially.
{\catcode`\|=\active
  \gdef\set#1{\mathinner{\lbrace\,{\mathcode`\|"8000\let|\midvert #1}\,\rbrace}}
  \gdef\Set#1{\left\{\:{\mathcode`\|"8000\let|\SetVert #1}\:\right\}}}
\def\midvert{\egroup\mid\bgroup}
\def\SetVert{\egroup\;\mid@vertical\;\bgroup}

% If the user is using e-TeX with its \middle primitive, use that for
% verticals instead of \vrule.
%
%\begingroup
% \edef\@tempa{\meaning\middle}
% \edef\@tempb{\string\middle}
%\expandafter \endgroup \ifx\@tempa\@tempb
% \def\mid@vertical{\middle|}
%\else
% \let\mid@vertical\vrule
%\fi
 %added by jjg

% by jjg \newcommand{\bra}[1]{\left\langle{#1}\right\vert}
% by jjg \newcommand{\ket}[1]{\left\vert{#1}\right\rangle}
    % Defines Dirac notation.
\newcommand{\qw}[1][-1]{\ar @{-} [0,#1]}
    % Defines a wire that connects horizontally.  By default it connects to the object on the left of the current object.
    % WARNING: Wire commands must appear after the gate in any given entry.
\newcommand{\qwe}[1][-1]{\ar @{{|}-} [0,#1]}
\newcommand{\eqw}[1][-1]{\ar @{-{|}} [0,#1]}
    % Defines a wire that terminates, on the right, with a -| (for "earth", or junk)
    % By default it connects to the object on the left of the current object.
    % Added by jjg
\newcommand{\qxT}{\ar@{-}[1,1]\qw} %Jons swaps top, leave blank after
\newcommand{\qxB}{\ar@{-}[-1,1]\qw}%Jons swaps bottom, leave blank after
\newcommand{\qwx}[1][-1]{\ar @{-} [#1,0]}
    % Defines a wire that connects vertically.  By default it connects to the object above the current object.
    % WARNING: Wire commands must appear after the gate in any given entry.
\newcommand{\cw}[1][-1]{\ar @{=} [0,#1]}
    % Defines a classical wire that connects horizontally.  By default it connects to the object on the left of the current object.
    % WARNING: Wire commands must appear after the gate in any given entry.
\newcommand{\cwx}[1][-1]{\ar @{=} [#1,0]}
    % Defines a classical wire that connects vertically.  By default it connects to the object above the current object.
    % WARNING: Wire commands must appear after the gate in any given entry.
\newcommand{\gate}[1]{*+[F]{#1} \qw}
    % Boxes the argument, making a gate.
\newcommand{\measure}{*++=[o][F]{M} \qw}
    % Inserts a measurement bubble.
    % BUG: Boxes made with \gategroup act as though the \measure graphic were zero size.
\newcommand{\measurebub}[1]{*+[F-:<.9em>]{#1} \qw}
    % Inserts a measurement bubble with user defined text.
\newcommand{\measuretab}[1]{*{\xy *+[F]{#1}\save;p,+LD;+LC-<.5em,0em> *{} **\dir{-}\restore\POS;p,+LU;+LC-<.5em,0em> *{} **\dir{-} \endxy} \qw}
    % Inserts a measurement tab with user defined text.
\newcommand{\multimeasure}[2]{*+{\hphantom{#2}} \qw \POS[0,0].[#1,0] !C *{#2} \POS[0,0].[#1,0] \drop+<0em,1em>!<0em,-.5em>\frm<.9em>{-}}
    % Draws a multiple qubit measurement bubble starting at the current position and spanning #1 additional gates below.
    % #2 gives the label for the gate.
    % You must use an argument of the same width as #2 in \ghost for the wires to connect properly on the lower lines.
\newcommand{\control}{*-=-{\bullet}}
    % Inserts an unconnected control.
\newcommand{\controlo}{*-=[o][F]{\phantom{\bullet}}}
    % Inserts a unconnected control-on-0.
\newcommand{\ctrl}[1]{\control \qwx[#1] \qw}
    % Inserts a control and connects it to the object #1 wires below.
\newcommand{\ctrlo}[1]{\controlo \qwx[#1] \qw}
    % Inserts a control-on-0 and connects it to the object #1 wires below.
\newcommand{\targ}{*{\xy{<0em,0em>*{} \ar @{ - } +<.4em,0em> \ar @{ - } -<.4em,0em> \ar @{ - } +<0em,.4em> \ar @{ - } -<0em,.4em>},*+<.8em>\frm{o}\endxy} \qw}
    % Inserts a CNOT target.
\newcommand{\qswap}{*=<0em>{\times} \qw}
    % Inserts half a swap gate.
    % Must be connected to the other swap with \qwx.
\newcommand{\multigate}[2]{*+{\hphantom{#2}} \qw \POS[0,0].[#1,0] !C *{#2} \POS[0,0].[#1,0] \drop\frm{-}}
    % Draws a multiple qubit gate starting at the current position and spanning #1 additional gates below.
    % #2 gives the label for the gate.
    % You must use an argument of the same width as #2 in \ghost for the wires to connect properly on the lower lines.
\newcommand{\ghost}[1]{*+{\hphantom{#1}} \qw}
\newcommand{\ghostX}[1]{*+{\hphantom{#1}}}
    % Leaves space for \multigate on wires other than the one on which \multigate appears.  Without this command wires will cross your gate.
    % #1 should match the second argument in the corresponding \multigate.
\newcommand{\push}[1]{*{#1}}
\newcommand{\pushT}[1]{*\txt{\tiny{$#1$}}} %jjg - A tiny push
\newcommand{\pushS}[1]{*\txt{\small{$#1$}}} %jjg - A small push
    % Inserts #1, overriding the default that causes entries to have zero size.  This command takes the place of a gate.
    % Like a gate, it must precede any wire commands.
    % \push is useful for forcing columns apart.
    % NOTE: It might be useful to know that a gate is about 1.3 times the height of its contents.  I.e. \gate{M} is 1.3em tall.
    % WARNING: \push must appear before any wire commands and may not appear in an entry with a gate or label.
\newcommand{\gategroup}[6]{\POS"#1,#2"."#3,#2"."#1,#4"."#3,#4"!C*+<#5>\frm{#6}}
    % Constructs a box or bracket enclosing the square block spanning rows #1-#3 and columns=#2-#4.
    % The block is given a margin #5/2, so #5 should be a valid length.
    % #6 can take the following arguments -- or . or _\} or ^\} or \{ or \} or _) or ^) or ( or ) where the first two options yield dashed and
    % dotted boxes respectively, and the last eight options yield bottom, top, left, and right braces of the curly or normal variety.
    % \gategroup can appear at the end of any gate entry, but it's good form to pick one of the corner gates.
    % BUG: \gategroup uses the four corner gates to determine the size of the bounding box.  Other gates may stick out of that box.  See \prop.
\newcommand{\rstick}[1]{*!L!<-.5em,0em>=<0em>{#1}}
    % Centers the left side of #1 in the cell.  Intended for lining up wire labels.  Note that non-gates have default size zero.
\newcommand{\rsticks}[1]{*!L!{#1}}
\newcommand{\lstick}[1]{*!R!<.5em,0em>=<0em>{#1}}
    % Centers the right side of #1 in the cell.  Intended for lining up wire labels.  Note that non-gates have default size zero.
\newcommand{\lsticks}[1]{*!R!{#1}}
\newcommand{\ustick}[1]{*!D!<0em,-.5em>=<0em>{#1}}
    % Centers the bottom of #1 in the cell.  Intended for lining up wire labels.  Note that non-gates have default size zero.
\newcommand{\dstick}[1]{*!U!<0em,.5em>=<0em>{#1}}
    % Centers the top of #1 in the cell.  Intended for lining up wire labels.  Note that non-gates have default size zero.
\newcommand{\ulstick}[1]{*!D!<.5em,-.5em>=<0em>{#1}}
\newcommand{\urstick}[1]{*!D!<-.5em,-.5em>=<0em>{#1}}
\newcommand{\Qcircuit}{\xymatrix @*=<0em>}
    % Defines \Qcircuit as an \xymatrix with entries of default size 0em.

% braket.sty          Macros for Dirac bra-ket <|> notation and sets {|}
% Donald Arseneau     asnd@triumf.ca     Last modified 27-Jan-2003.
% This is free, unencumbered, unsupported software.
%
% Commands defined are:
% \bra{ }   \ket{ }   \braket{ }   \set{ }    (small versions)
% \Bra{ }   \Ket{ }   \Braket{ }   \Set{ }    (expanding versions)
% 
% The "small versions" use fixed-size brackets independent of their
% contents, whereas the "expanding versions" make the brackets and 
% vertical lines expand to envelop their contents (internally using 
% the \left and \right commands).  You should use the vertical bar
% character "|" to input any extra vertical lines.  In \Braket these
% vertical lines will expand to match the arguments, and in \Set the
% first vertical will expand this way.  E.g.,
%   \Braket{ \phi | \frac{\partial^2}{\partial t^2} | \psi }
%   \Set{ x\in\mathbf{R} | 0<{|x|}<5 }
%
% NOT defined is "\ketbra" (for projection operators) because I prefer
% \ket{ } \bra{ }.
%
% Because each definition is so small, it makes no sense to have a 
% complicated generic version for many bracket styles.  Instead, 
% you can just copy the definitions and change \langle or \rangle,
% < and > to what you like.
%
\def\bra#1{\mathinner{\langle{#1}|}}
\def\ket#1{\mathinner{|{#1}\rangle}}
\def\braket#1{\mathinner{\langle{#1}\rangle}}
\def\Bra#1{\left<#1\right|}
\def\Ket#1{\left|#1\right>}
{\catcode`\|=\active 
  \gdef\Braket#1{\begingroup \mathcode`\|32768\let|\BraVert\left<{#1}\right>\endgroup}
}
\def\BraVert{\egroup\,\mid@vertical\,\bgroup}
% The \mid@vertical is \vrule with ordinary TeX but \middle| in eTeX.
% We always avoid a \mathchoice in making the inner vertical lines.  
% Note that \right>, prints the same as \right\rangle but is faster.  
%
% \def\ketbra#1#2{\ket{#1}\bra{#2}}
% \def\Ketbra#1#2{\left|{#1}\vphantom{#2}\right>\left<{#2}\vphantom{#1}\right|}

% \Set{...|...} Only the first | is treated specially.
{\catcode`\|=\active
  \gdef\set#1{\mathinner{\lbrace\,{\mathcode`\|"8000\let|\midvert #1}\,\rbrace}}
  \gdef\Set#1{\left\{\:{\mathcode`\|"8000\let|\SetVert #1}\:\right\}}}
\def\midvert{\egroup\mid\bgroup}
\def\SetVert{\egroup\;\mid@vertical\;\bgroup}

% If the user is using e-TeX with its \middle primitive, use that for
% verticals instead of \vrule.
%
%\begingroup
% \edef\@tempa{\meaning\middle}
% \edef\@tempb{\string\middle}
%\expandafter \endgroup \ifx\@tempa\@tempb
% \def\mid@vertical{\middle|}
%\else
% \let\mid@vertical\vrule
%\fi

\usepackage{eepic}
\usepackage{times} \usepackage{mathptm}  
\usepackage{amssymb}
\usepackage{latexsym}
\usepackage{url}
\RequirePackage{vmargin}
\setmarginsrb{1.9cm}{2cm}{1.9cm}{2cm}{0pt}{0pt}{0pt}{0pt} %jg 2cm all round
  %ctm shaves another mil left and right
\setlength{\columnsep}{1cm}

%jg redefined title layout
\makeatletter
\def\maketitle{%
\begin{flushleft}
\LARGE\bfseries \@title \bigskip \\
\Large\mdseries \@author \medskip
\end{flushleft}%
}
\makeatother

\setlength{\marginparwidth}{1cm}
\setlength{\marginparsep}{3mm}
\def\marginnote#1{\marginpar{\scriptsize\itshape\raggedright#1}}

%%%\renewcommand{\baselinestretch}{0.9641}\small\normalsize
% \renewcommand{\baselinestretch}{0.972373}\small\normalsize %jg ick!
\makeatletter
%%%\setlength{\parskip}{0\p@ \@plus 0.98\p@}
% \setlength{\parskip}{0.75\parskip} %jg prefer a parindent
\renewcommand{\section}{\@startsection
 {section}%
 {1}%
 {0mm}%
 {-\baselineskip}%
 {0.5\baselineskip}%
 {\Large\bfseries}%
 }%
\renewcommand{\subsection}{\@startsection
 {subsection}%
 {2}%
 {0mm}%
 {-0.75\baselineskip}%
 {0.375\baselineskip}%0.01
 {\large\bfseries}%
 }%
\makeatother

\pagestyle{empty}

% \parskip 0.05in \parindent 0in %jg prefer a parindent
\newcommand{\sparadrap}[1]{\par\mbox{\textbf{#1}$\;\;$}}
\newcommand{\squashit}{\itemsep=0pt
  \topsep=0pt \parskip=0pt \parsep=0pt}
\let\ITEMIZE\itemize \def\itemize{\ITEMIZE\squashit}

\newcommand{\remph}{\emph}
\newcommand{\demph}{\textbf}

\newcommand{\bsize}[2]{{#1\textbf{#2}\par}}
\newcommand{\bL}{\vspace{0.1in}\noindent\bsize{\Large}}
\newcommand{\bl}{\bsize{\large}}
\newenvironment{titemize}{
  \begin{list}{\ensuremath{\bullet}}{
  \setlength{\topsep}{1pt}
  \setlength{\parsep}{0pt}
  %\setlength{\parskip}{0pt}
  \setlength{\itemsep}{1pt}
  \setlength{\partopsep}{0pt}}
}
{
  \end{list}
}

\usepackage{lscape}
\usepackage{pstricks}
\usepackage{pst-grad}
%%\usepackage{pst-xkey}
\usepackage{multido}

\makeatletter
\makeatother

\newcommand{\surl}[1]{\small{\url{#1}}}
\newcommand{\parametris}{parametriz}





\title{Quantum Computing Meets Type Theory:\\
 \Large A Framework for Certified Quantum Information Processing
% \\
% \LARGE (Case for Support)
}
\author{(Case for Support) --- Thorsten Altenkirch and Alexander S. Green}
\date{}

\begin{document}

%\twocolumn[
\maketitle
\section*{Summary}
Quantum Information Processing is an emerging technology with an enormous
potential: while hypothetical quantum computers can solve certain
computational problems exponentially faster than conventional
computers \citemain{shor, simon},
cryptographic protocols based on quantum keys \citemain{BEN84,e91} are
already used to provide a previously unknown level of security
\citemain{qkd}. Researchers in
computer science and related disciplines are facing the challenge of
developing quantum algorithms \citemain{cacm} and protocols \textbf{and}
providing evidence that their quantum software performs as
specified. With the Quantum IO (QIO) monad \citetrack{txa:qio,asg:thesis} we have
presented a framework for the development of quantum software within
the functional programming language Haskell. Functional
Programming is ideally suited to meet the challenge of quantum
computing due to its ability to model
non-standard effects. However, some aspects of Quantum Information Processing,
in particular preservation of reversibility, cannot be modelled satisfactorily
within Haskell's conventional type system. In his PhD work, Green shows
that this issue can be addressed by moving to a more expressive type
system, i.e. the dependently typed programming language Agda \citemain{agda}.

This is only the first glimpse of the role dependent types could play
for Quantum Information Processing --- there is much more to come!
Quantum information processing is a complex topic and needs expressive
notations for us to use it. Dependent types give rise to very expressive
languages, which are extremely useful for specifying and certifying systems exactly.
%{\color{red}evidence?}. 
This makes them ideally suited to helping us to tackle the
problem of specifying and implementing devices for quantum information
processing. 

% This is the core of Alexander S. Green's PhD
% work \citemain{asg:thesis} in which he also addresses some key issues with
% this approach which are related to the fact that Haskell's type system
% is too weak to specify the invariants of reversible quantum
% computations. He also suggests a way to address this problem by moving
% to a stronger type system as realised within the dependently typed
% functional programming language Agda \citetrack{agda}. 

We propose to explore and validate this claim by developing a
framework for formally certified Quantum Information Processing within a
dependently typed programming system. Within this framework we plan to
certify existing and emerging quantum algorithms and protocols \textbf{and}
provide certified compilers to low level, physically plausible
realisations of quantum computing such as measurement based quantum
computing %\citemain{oneway,meascalc}.
\citemain{oneway}.  The formal verification of 
quantum protocols is especially relevant since applications of this
technology are becoming commercially available \citemain{qci}.

We will utilise this framework to develop and certify high level
approaches to Quantum Information Processing such as Abramsky's,
Coecke's and Selinger's graphical calculi %\citemain{ac1,ac2,ac3,abr,sel}. 
\citemain{ac1,seldccc}. 
We envisage
that our framework will provide the language and software tools for
other researchers to use in their work on high level approaches to
Quantum Information Processing. Indeed, our proposal naturally
complements the development of more specific calculi for Quantum
Information Processing, which focus on automatisation while our
strength is expressivity.% \citemain{quanto,upro}.

Our proposal is for a four year project with Alexander S. Green as the
RA and one PhD student.

% Quantum computers have been shown to offer a possible exponential
% speed up over certain classical algorithms. For example, Shor's
% algorithm\citemain{shor} for finding the prime factors of large composite
% numbers. However, the process of defining new quantum algorithms, or
% programs for these quantum machines, is currently left in the realms
% of the mathematics that are used to model the underlying quantum
% mechanical aspects. Research into quantum programming languages is
% necessary to propagate the techniques available when working with
% quantum data, into the mainstream world of computer science. Quite a
% few languages currently exist (see \citemain{RudigerSurvey,GaySurvey}),
% but fail to grasp the structure of the very quantum effects that give
% these computers their extra computational power. High-level quantum
% languages such as QPL\citemain{qpl},  and
% QML\citemain{txa:qml,jjgthesis,jjg:qml} have started to look at
% abstractions that can be used to model these quantum structures
% effectively. However, we believe a bottom-up approach is equally
% suited to this task, and have been developing a low-level interface to
% quantum computation that uses the well studied structure of monads to
% give us explicit access to quantum effects. The work so far has
% introduced the Quantum IO Monad (QIO) \citemain{txa:qio,asg:thesis}, a
% library for writing quantum computations in the functional programming
% language Haskell \citemain{haskell}. However, further research is required
% to develop QIO into a fully developed quantum programming
% language. One development, that seems necessary for QIO, is for it to
% be developed in a dependently typed environment. Indeed, dependent
% types are required to ensure the unitarity of operations that can be
% defined. A little work has already been done on reimplementing QIO in
% the dependently typed language Agda \citemain{agda}, but this is only the
% tip of the iceberg for developing a formally verified functional
% quantum programming language. 

% Based on the previous work at Nottingham on the Quantum IO Monad, we
% plan to explore how QIO can be further developed, with the ultimate
% goal of producing QIO as a standalone quantum programming language,
% that uses a monadic structure to model quantum effects, as well as a
% dependent type system that leads to formally verified unitary
% structures. We seek funding for an RA (Alexander S. Green, whose PhD
% laid much of the groundwork for this proposal \citemain{asg:thesis}), and
% a doctoral student, both here at Nottingham. As well as the
% appropriate support for equipment and travel.

%
%]

%\newpage

\section*{Part 1: Track Record}

\subsection*{Thorsten Altenkirch}
Thorsten Altenkirch received his PhD from the University of
Edinburgh in 1993. He has been a research assistant at the University of
Edinburgh and at Chalmers University in Gothenburg, Sweden and has
held lecturing positions at the University of Munich, Germany, and the
University of Nottingham. Since October 2006 he has been a Reader in
Computer Science at the University of Nottingham and in October
2008 he founded the Functional Programming Laboratory together with
Graham Hutton. Altenkirch and Hutton are now jointly leading the
laboratory.

Altenkirch's research interests are in Type Theory, Category Theory, Functional
Programming and Quantum Programming. 
Among his main achievements is
the first formalised normalisation proof for System F
\citetrack{alti:tlca93}
, his work on normalisation by evaluation 
\citetrack{alti:lics01}%,txa:jtait}
, extensional equality within intensional Type Theory \citetrack{alti:lics99}
, the development of container theory \citetrack{alti:cont-tcs}%,alti:jpartial}
, on
monads \citetrack{alti:fossacs10} 
and on generic programming 
\citetrack{alti:wcgp02}
.
Especially relevant for this proposal is the
work with Green on the Quantum IO Monad \citetrack{txa:qio,txa:qpl06}
, Grattage on QML \citetrack{txa:qml} and with Vizzotto and Sabry on
modelling quantum computation \citetrack{txa:qarrows,txa:qpl05} using
arrows. Altogether, he has published over 40 papers --- for a full list please
consult \texttt{http://www.cs.nott.ac.uk/\~{}txa/publ/}.

He has been on the program committee of a number of conferences and
workshops, recently CiE 2008, TPHOLs 2009 and ITP 2010. He has
organised TYPES meetings in 1998 and 2006 and has coedited the
proceedings; and he has also been the program co-chair of PLPV 2009.  He
organised a Dagstuhl seminar on dependently typed programming in 2004
and has since organised two further workshops on this topic (DTP 2008
and DTP 2010). He gave an invited lecture series on Type Theory at the
University of Tallinn in 2003 and he was a visiting Professor at the
Universit\`{e} Denis Diderot, Paris in  July 2007. He gave invited talks
at the first workshop on quantum programming languages (QPL 2003), the
workshop on normalisation by evaluation (NBE 2009) and at the special
session on proof theory of CiE 2010.

\twocolumn[

He has been the Principal Investigator on two successfully completed
EPSRC grants : \emph{Modelling Irreversible Quantum Computation}
(GR/S30818/01), \emph{Observational Equality For Dependently Typed
  Programming} (EP/C512022/1); and has been co-investigator on
\emph{Theory and Applications Of Containers} (EP/C511964/1). 
He is currently the principal investigator of two ESPRC
projects: \emph{Reusability and Dependent Types} (EP/G034109/1), 
networked with
Oxford and Strathclyde, Nottingham is the lead site and 
\emph{Theory And Applications of Induction
  Recursion} (EP/G03298X/1), networked with Strathclyde and Swansea. 
On the European level he has been active in the the
TYPES and APPSEM coordination actions and has been involved in the
QICS STREP. He also hosted a Marie-Curie fellowship (2006 - 2008). He
is regularly teaching courses at the Midland Graduate School and has
been the co-investigator of the associated EPSRC grant: \emph{Midlands
  Graduate School in the Foundations of Computer Science}
(GR/T06087/01). He has supervised 5 successfully completed PhD
projects at Nottingham, two of them in the area of quantum
programming \citetrack{jjgthesis,asg:thesis}. He has also lectured 
on the subject of quantum computing at third year
undergraduate level, masters level, and to research students 
(at the Midlands Graduate School).

\subsection*{Alexander S. Green}
Alexander S. Green will be the researcher working full-time on the
project. His PhD thesis, \emph{Towards a formally verified functional
  quantum programming language} \citetrack{asg:thesis}, focused
specifically on defining the first approximations of QIO, in both
Haskell and Agda. Green's PhD was written under the supervision of
Thorsten Altenkirch, and he presented his work at BCTCS 2006
and 2007, QPL 2006, QNET workshops 2007; 2008; and 2009, QICS workshop
2008, TFP 2008, and FitA May 2009. His paper, \emph{From reversible to
irreversible computations} \citetrack{txa:qpl06}, appeared in the
proceedings of QPL 2006, and he co-authored a chapter on \emph{The
Quantum IO Monad}, which appears in the book \emph{Semantic Techniques
in Quantum Computation} \citetrack{txa:qio}. 

Since completing his PhD, Green has held a postdoctoral research
fellowship position at Dalhousie University, in Halifax, Nova Scotia,
Canada. His work there, as part of the QCS project, focused on
developing Quipper, a scalable quantum programming language. He
co-authored two papers on Quipper: \emph{Quipper: A Scalable Quantum
  Programming Language} \citetrack{quipper-pldi}, which appeared
in the proceedings of PLDI 2013, and \emph{An Introduction to Quantum
  Programming in Quipper} \citetrack{quipper-rc}, which as well as
appearing in the proceedings of RC2013, was also presented by Green at
the workshop.

Green has also lectured third year undergraduate level and masters
level courses on quantum computation in the School of Computer
Science, at The University of Nottingham. 

\subsection*{Nottingham - host organisation}

The School of Computer Science at the University of Nottingham
is a research-led School in one of the leading Universities in
the UK.	 The School was ranked 8th in the last Research Assessment
Exercise, and the Functional Programming Lab within the School is
one of four major research groups, with an international reputation
for its work on formally-based approaches to software construction
and verification.  The FP lab currently comprises 4 academic staff
(Thorsten Altenkirch, Venanzio Capretta, Graham Hutton, and Henrik
Nilsson), 2 postdoctoral fellows, and 9 PhD students.  To date the
group has received \pounds 1.5M of EPSRC funding over 14 projects,
and has 12 completed PhD students.

The Functional Programming Lab provides a highly stimulating
research environment for researchers and PhD students with weekly
research meetings and frequent seminars. % Our research blog
% %\citetrack{fplunchweb} 
% \texttt{http://sneezy.cs.nott.ac.uk/fplunch/weblog/} 
% is illustrative of the lively and exciting
% research environment at Nottingham.

% Together with Edinburgh,
% Cambridge, Oxford and York, Nottingham is one of the leading
% institutions for functional programming within the UK and given the
% strength of functional programming in the UK, worldwide.
% %%{\color{red}
% %%(how much?) 
% %%}

% {\color{red} Add data: how much EPSRC money, how many PhD thesis
%   completed, and any other evidence.}

% The laboratory
% comprises Thorsten Altenkirch, Venanzio Capretta, Graham Hutton and 
% Henrik Nilsson as
% academic members of staff, and has currently 2 research assistants and 10
% PhD students.  

%\twocolumn
\vspace{1em}
]




{\small 
\bibliographystyletrack{abbrv}
\bibliographytrack{proposal} 
}

%\newpage

\newpage
% \null
% \newpage
% \null
% \newpage

\section*{Part 2: Proposed Research}

\section{Background}

\subsection*{Quantum Computing}
Quantum computing is based on the realisation that computation based
on quantum physics is fundamentally different than computation based
on a classical model of the world. It was first observed by Feynman 
\citemain{feynman81}
that quantum processes cannot be efficiently modelled on a classical
computer. This lead to the proposal of quantum computing as a
fundamental notion. Famously, Shor showed that factorisation can be
done in polynomial time on a hypothetical quantum computer
\citemain{shor} while the
best known classical algorithm needs exponential time
\citemain{sieves}. This entails
that a quantum computer would represent a threat to public key
cryptography based on RSA \citemain{rsa} and other algorithms. However, quantum
physics also provides a solution to this problem using secure quantum
keys based on the no-cloning theorem which states that it is
impossible to construct a precise copy of a given quantum
state. Introductions to quantum computing for computer scientists are
now widely available (\citemain{NC, mermin,cacm}) and the
subject is being taught in advanced undergraduate courses
such as the one at Nottingham (based on \citemain{mermin} and using QIO). 

General purpose quantum computers are still only a hypothetical
device. However, researchers in physics are making progress on these
issues, exploiting a number of possible physical realisations of quantum
computing \citemain{sussex,shorimpl}. The key to a successful
implementation may not only lie in the realm of hardware but could
also exploit different low level approaches to quantum computing, such
as the measurement based approach \citemain{oneway,meascalc} 
which is an alternative to the more
conventional approach based on quantum circuits.

Current work in the field of quantum computing is progressing in a number 
of areas:
\begin{itemize}
\item
Researchers are trying to identify problems and find quantum algorithms which
offer a substantial speedup in comparison to classical approaches
\citemain{cacm}, e.g. it is currently an open problem whether the graph
isomorphism problem can be solved efficiently on a quantum computer
\citemain{graph}.  
\item
%%{\color{red}
Quantum cryptographic protocols such as BB84 \citemain{BEN84} and E91
\citemain{e91} have been developed to provide unconditionally secure
distribution of keys, that can then be used to provide secure
communication over classical channels. Systems are being developed
that can actually implement these protocols over networks of optic
fibre. Recent work includes \citemain{qkd} which introduces an
implementation that has been tested over 184.6km of optic fibre, 
and researchers at Toshiba \citemain{toshiba} have also
demonstrated a system that is able to continuously distribute keys
over a 36 hour period at 1 Mbit/s. 
\item A number of quantum programming languages have been proposed,
  such as QPL \citemain{qpl}, inspired by flowcharts, and QML
  \citetrack{txa:qml,jjgthesis} a first order functional quantum
  programming language. This languages offer specific constructs which
  make it easier to express quatum algorithms on a high level.
  Quantum process calculi \citemain{upro} have been proposed to capture
  the nature of concurrent and communicating quantum systems.  Recent
  surveys on the state of the art for quantum programming languages
  can also be found \citemain{RudigerSurvey,GaySurvey}.
%%} 
\item
Researchers are trying 
to get a better understanding of the
high level structure of Quantum Information Processing. A promising approach here
is the work by Abramsky, Coecke and Selinger \citemain{ac1,seldccc}, 
which exploit
categorical models of quantum systems, i.e. dagger compact closed
categories \citemain{seldccc}, to develop a graphical calculus for reasoning about
quantum systems \citemain{kgqm}. Extensions of this calculus have already
been used to verify some basic quantum protocols \citemain{perdrix}.
\end{itemize}

% Quantum computation makes use of the physical properties that appear
% when looking at the behaviour of particles at their lowest
% level. The idea of quantum computing can be tracked back to a talk by
% Richard P. Feynman \citemain{feynman81} on simulating physics with
% computers, showing that a classical computer is unable to efficiently
% model quantum systems.
% The state of a particle can be described by a wave function
% that collapses to a classical value when we observe the particle. In
% computer science, we often think of things in terms of bits, and it is
% therefore natural to define a quantum bit (or qubit) as a quantum
% particle that can be described by a wave function over two orthogonal
% states (which we shall denote $\ket{0}$ and $\ket{1}$). The state of
% an arbitrary qubit can be given by $\alpha\ket{0} + \beta\ket{1}$,
% where $\alpha, \beta \in \mathbb{C}$ (such that $| \alpha |^2 + | \beta
% |^2 = 1$) define the amplitudes of each base state within an overall
% superposition of both. A measurement of a qubit will leave us with the
% state $\ket{0}$ with probability $| \alpha |^2$ or $\ket{1}$ with
% probability $| \beta |^2$. It is when we start to look at multiple
% qubit \emph{systems} that we can see even more interesting
% behaviour. A system of $n$ qubits can exist in a linear superposition
% of all $2^n$ bit strings of length $n$, and thus the \emph{state size}
% grows exponentially with the number of qubits. States can become
% \emph{entangled}, whereby the state of each qubit within a system can
% not be described fully on its own, but yet it can still be treated as
% an individual particle. It is these entangled states, that when used
% effectively within a computation, can give rise to the exponential
% speed up over classical solutions. It is a slight over-simplification,
% but you can think of a quantum computation as running in parallel over
% all the states in a superposition. However, when a system is measured,
% it will collapse into a single base state. This means that measuring a
% single qubit within a larger quantum system can have side effects that
% change the state of other qubits with which it was entangled.

\subsection*{Functional Programming}
Functional programming is based upon the idea of programs being
functions from arguments to results, rather than operations that
modify stored values.  But it is much more than just another
programming paradigm, it creates a bridge between abstract reasoning and
concrete programming. Thus it provides the necessary mathematical tools
to put software engineering on solid foundations --- with the aim of
repeating the success of mathematical methods in conventional
engineering.  The research carried out at the Functional Programming
Laboratory in Nottingham reflects this broad vision:
\begin{itemize}
\item We are exploring functional notions of effects in the design and
  application of modelling and simulation languages, and in reasoning
  about various forms of effects including state, exceptions,
  interrupts, concurrency, and also quantum effects
  \citemain{swierstra-altenkirch:beast,wss} .
\item Inspired by the Curry-Howard equivalence, we are investigating
  novel languages based upon dependent types, with the aim of bringing
  programming and verification closer together \citemain{morris:thesis,pisig}.
\item We are exploiting mathematical concepts such as monads and
  arrows that arise in category theory to find new ways to structure
  and reason about programs \citemain{mauro:thesis,george}.
\end{itemize}
Imperative languages exploit side-effects based on the concept of an
updatable memory. This notion of effects is also available in
functional programming languages like Haskell, via the notion of a
monad, here Haskell's IO monad. However, this notion of an effect is
not hardwired in the core of the language. As a consequence functional
programming languages are much more flexible when it comes to realise
novel notions of effects such as transaction based concurrency 
\citemain{stm},
or as we have shown recently for Quantum Information Processing \citetrack{txa:qio,asg:thesis}.
At the same time, functional implementations of effects can be
understood as providing a constructive mathematical semantics of
various effects and hence provide a base for reasoning about them
\citemain{swierstra-altenkirch:beast,wss}. 
This is not just a hypothetical suggestion
but indeed exploiting dependently typed languages such as Agda we have
used this to reason about multi-core algorithms \citemain{alti:tfp08}.  

% Functional programming gives us a mathematically based way of defining
% computations as pure functions. This gives rise to the problem that in
% computation, there are many things that we want to do that can't be
% modelled as pure functions. For example, any sort of I/O is impure, as
% the behaviour of a function can depend on more than just the
% arguments given to the function. However, the categorical notion of a
% Monad can be used to model this impure behaviour. The type of a
% function can then describe the side effects that may take place when
% the function is called. In Haskell \citemain{haskell}, for example, all
% I/O takes place using the IO Monad. Having a monadic structure means
% that any program which makes use of impure functions must do so
% explicitly. It has also been shown how the design of a monadic
% structure can give a sound base for reasoning about these effectful
% computations \citemain{swierstra-altenkirch:beast,wss}.
% This leaves the area of functional programming in a unique
% place to define quantum programming languages. The side effects that
% can be caused by measurements in a quantum computation can be
% described explicitly by using a monadic structure.

\subsection*{The Quantum IO Monad in Haskell}

\begin{figure}
\begin{pspicture}(-1,-1)(7,7)

\psframe[framearc=0.3](0.1,0)(2.6,2.5)
\rput(1.35,2.0){\psscaleboxto(2.2,0){U Monoid}}
\rput(1.35,1.2){\psscaleboxto(2,0){reversible}}
\rput(1.35,0.4){\psscaleboxto(2,0){deterministic}}

\rput(1.35,3.3){\psline[linewidth=2pt]{->}(0,-0.8)}
\pstextpath{
\rput(1.65,3.3){\psline[linewidth=1pt,linearc=0.8,linecolor=white]{->}(0,-0.8)}
}{
Run
}

\psframe[framearc=0.3](0,3.3)(2.7,6)
\rput(1.35,5.55){\psscaleboxto(2.4,0){QIO Monad}}
\rput(1.35,4.65){\psscaleboxto(2,0){irreversible}}
\rput(1.35,3.75){\psscaleboxto(2,0){probabilistic}}

\psline[linewidth=2pt,linearc=0.8]{->}(2.7,4.3)(3.5,4.6)(4.5,4.25)
\pstextpath{
\psline[linewidth=1pt,linearc=0.8,linecolor=white]{->}(2.7,4.6)(3.5,4.9)(4.5,4.55)
}{
Initialisation
}

\psline[linewidth=2pt,linearc=0.8]{<-}(2.1,3.3)(3.1,2.9)(4,3.6)
\pstextpath{
\psline[linewidth=1pt,linearc=0.8,linecolor=white]{-}(2.1,3.0)(3.1,2.7)(4,3.3)
}{
Measurement
}


\psline[linewidth=2pt]{<->}(2.6,1.6)(4,2.5)
\pstextpath{
\psline[linewidth=1pt,linearc=0.8,linecolor=white]{-}(2.7,1.3)(4.1,2.2)
}{
Modifies
}


\psframe[shadow=true,shadowsize=8pt](4,1.75)(6,4.25)
\rput(5,3.5){\psscaleboxto(1.2,0){Qubit}}
\rput(5,2.5){\psscaleboxto(1.2,0){Store}}

\end{pspicture}
\caption{Structures used in the Quantum IO Monad}\label{fig:qio-haskell}
\end{figure}

The Quantum IO Monad \citetrack{txa:qio,asg:thesis} is defined as a
Haskell library that acts as an interface to quantum computation
(figure \ref{fig:qio-haskell}). It uses a monadic structure that gives
explicit access to the effectful nature of measurements in quantum
computation.  Computations in QIO usually consist of an initialisation
stage, a unitary transformation, i.e. a reversible computation, and a
measurement stage. The unitary transformations (the U monoid, actually
a group) are by their very nature, pure, and as such can be defined
separately from the monadic structures of QIO. The implementation
allows unitary operators to contain conditional structures, following
the ideas of quantum data and control introduced by QML
\citetrack{txa:qml}, as well as ancillary qubits that can be used
within a unitary structure, following from the ideas of heap and
garbage introduced in \citetrack{txa:qpl06}. These features
necessitate side conditions that can only be checked at run-time to
ensure unitarity.
% , such as making sure an ancillary qubit
% is returned to its initial state by the unitary structure that uses
% it. 
A stronger type system, i.e. a dependently typed language, makes it
possible to provide evidence so that these side conditions are
checked at compile time, by the type checker. The current Haskell
implementation of QIO is available through Hackage,
and has been used in third year undergraduate level and masters level
courses on quantum computation in the School of Computer Science, at
The University of Nottingham.

\subsection*{Dependent Types}
Dependent types are types which refer to (hence depend on)
data. A typical example is the type of vectors: we write
$\mathrm{Vec}\,A\,n$ for the type of length $n$ sequences of type $A$
items. Vectors refine the ubiquitous types of lists $[A]$.
Conventionally, a program $\mathrm{idM}$ computing the identity
matrix, represented as a sequence of sequences of real numbers, of a
given finite dimension would have the type $\mathrm{idM} : \Bbb{N} \to
[[\Bbb{R}]]$. In a dependently typed language the same program could
be given the more informative type $\mathrm{idM} : (n:\Bbb{N}) \to
\mathrm{Vec}\,(\mathrm{Vec}\,\Bbb{R}\,n)\,n$, clearly indicating the
relationship between the value of the input and the structure of the
output; and documenting the fact that $\mathrm{idM}$ always returns
square matrices.

Dependent types have their origin in the constructive foundations of
mathematics developed by the Swedish
philosopher and mathematician Per Martin-L\"of, who showed that the
\emph{Curry-Howard} isomorphism between simple types and propositional
logic naturally extends to an isomorphism between dependent types and
higher-order constructive logic.  Exploiting this equivalence between
logical propositions and types, dependent types have been used as the
basis for proof assistants --- the most advanced system developed in
Europe, Coq, has been used to formalise Mathematics (e.g.\
the proof of the four colour theorem
\citemain{gonthier:four-colour-paper}) and to verify programs (e.g.\
correctness of a C
compiler \citemain{compcert-back}).

The language Agda \citemain{agda}, developed by Ulf Norell, is a functional
programming language based on dependent types, enabling programmers to
express strong invariants by using its expressive type system. At the
same time Agda is an interactive theorem prover, like Coq, by simply
exploiting the Curry Howard equivalence. The powerful idea of using
reflection, which is also the base for the formalisation of the 4
colour theorem in Coq, provides opportunities for essential
automatisation of trivial sub proofs. Agda has been very well received
by the community due to Norell's pragmatic approach to tool
development (see the Agda wiki)
and is increasingly used in many places 
(E.g. \citemain{agda4}).

In the programming language community there is an increasing interest
in dependently typed programming due to its expressiveness and its
potential to capture many, if not all, extensions to type
systems. Consequently there are a number of dependently typed systems
and languages under development such as Trellis \citemain{trellis}, Omega
\citemain{omega} and Epigram 2.

% The study of dependent types follows from Per Martin-Lof's
% intuitionistic type theory \citemain{itt}.

% \subsection*{The Quantum IO Monad in Agda}
% The reimplementation of QIO in Agda \citemain{asg:thesis} uses the
% dependent type system to give formal proofs that the unitary
% structures that can be defined are indeed unitary. This is important
% for the conditional structures and the use of ancillary qubits, that
% were introduced in the original Haskell implementation. The
% implementation is only a first approximation, and is impeded by the
% current lack of an implementation of complex numbers in Agda. However,
% computations can be written in QIO, and compiled down to an underlying
% floating point representation of complex numbers, in which we must
% postulate many of the facts that would hold for a true implementation
% of the complex number field.

\subsection*{Example: using dependent types for QIO}
\label{sec:dependent-types-qio}

In the Haskell implementation of the QIO monad we provide an operation
\texttt{ifq} which allows us to run unitary quantum computations
conditionally, depending on the value of a qubit; without measuring it. 
The type of this operation is 
\begin{verbatim}
ifq : Qbit -> (Bool -> U) -> U  
\end{verbatim}
Here \texttt{Qbit} is the type of references to qubits in the quantum
memory and \texttt{U} is the type of unitary transformations. For
example
\begin{verbatim}
ifq x (\ b -> if b then uhad y 
                   else mempty)  
\end{verbatim}
will perform a Hadamard transform on qubit \texttt{y} conditionally 
depending on the value of qubit \texttt{x}. 
This corresponds to the following quantum circuit:

\[
\Qcircuit @C=1em @R=1em {
  \ket{x} & & \ctrl{1}        & \qw \\
  \ket{y} & & \gate{Hadamard} & \qw
}
\]


However, the typing of \texttt{ifq} is too imprecise to ensure that
unitarity is preserved. For example we can express the following
operation, which unconditionally resets the \texttt{x} qubit to $\ket{0}$:
\begin{verbatim}
ifq x (\ b -> if b then unot x 
                   else mempty)  
\end{verbatim}
This operation is certainly not unitary. In the current implementation
running such a program would lead to a runtime error. This is
certainly not 
satisfactory. In a dependently typed environment this issue can be
addressed by introducing a predicate
\begin{verbatim}
Sep : Qbit -> U -> Set
\end{verbatim}
where \verb+Sep x u+ is the set of proofs certifying that the qubit \texttt{x} is separable
from the unitary \texttt{u}. The type of \texttt{ifq} now becomes:
\begin{verbatim}
ifq : (x : Qbit) 
   -> (us : Bool -> U) 
   -> ((b : Bool) -> Sep x (us b)) 
   -> U  
\end{verbatim}
Here we use depdendent function types \verb+(x : A) -> B x+ to 
avoid any possible runtime errors by requiring a proof
that the qubit we branch over is separable from both
branches. Moreover this fact is formally certified in the
implementation. For the first instance we can easily prove that
\texttt{x} is separable from both branches because it doesn't occur
them (this is a sufficient but not a necessary condition). In the 2nd
case this is not provable since \texttt{x} is not separable from
\verb+unot x+.

\section{Programme and Methodology}

Our approach is driven by the development of the framework: first
implementing the QIO framework (figure \ref{fig:qio-fw}) within a dependently typed system and
then using and extending this framework to certify known and novel
quantum algorithms and protocols. We also plan to use and extend this
framework in two directions: on the back-end we aim to explore compilation
into quantum hardware and on the front-end we look at the
representation of high level structures within our framework. The
certified compilation of quantum algorithms would be ideally suitable
as a PhD project. 

The success of our project depends very much on the interaction with
our colleagues in the area of quantum information processing. We plan
to exploit contacts which have developed during the QNET project in
several work packages. To focus and develop these interactions we plan
to organise a a workshop related to the QIO framework in the middle of
the project.

To see wether current methods for developing certified programs are
applicable in quantum information processing is an important research
challenge. Providing answers here will feed back on not just the design of
reasoning formalisms, but also on the way we think about and approach
quantum structures.

\begin{figure}
\begin{pspicture}(-1,-1)(7,10)

\rput(1.6,6.1){\psline[linewidth=2pt]{<->}(0.7,1.6)}
\pstextpath{
\rput(1.3,6.3){\psline[linewidth=1pt,linecolor=white]{-}(0.84,1.92)}
}{
Reasoning
}

\rput(3.7,7.7){\psline[linewidth=2pt]{<->}(0.7,-1.6)}
\pstextpath{
\rput(4,8){\psline[linewidth=1pt,linecolor=white]{-}(0.84,-1.92)}
}{
Programming
}

\rput(1.6,2.9){\psline[linewidth=2pt]{<->}(0.7,-1.6)}
\pstextpath{
\rput(1.3,2.8){\psline[linewidth=1pt,linecolor=white]{-}(0.84,-1.92)}
}{
Simulation
}

\rput(3.7,1.3){\psline[linewidth=2pt]{<->}(0.7,1.6)}
\pstextpath{
\rput(4,1.1){\psline[linewidth=1pt,linecolor=white]{-}(0.84,1.92)}
}{
Compilation
}

\pscircle[linecolor=black,fillcolor=cyan,fillstyle=solid](3,8.5){1}
\rput(3,8.5){\psscaleboxto(1.3,0){User}}

\psframe[linecolor=black,fillcolor=cyan,fillstyle=solid](1,3)(5,6)
\rput(3,4.9){\psscaleboxto(2,0){QIO}}
\rput(3,4.1){\psscaleboxto(3,0){framework}}

\pscircle[linecolor=black,fillcolor=cyan,fillstyle=solid](3,0.5){1}
\rput(3,0.8){\psscaleboxto(1.2,0){Hardware}}
\rput(3,0.5){\psscaleboxto(0.2,0){or}}
\rput(3,0.2){\psscaleboxto(1.7,0){Physical Model}}

\end{pspicture}
\caption{Structure of the QIO framework}\label{fig:qio-fw}
\end{figure}

Based on the summary above, we arrive at the following work packages:

\subsection*{WP1 : The QIO framework in a dependently typed system}
\label{sec:wp:qio}

In the final chapters of his PhD thesis, Green develops the basic
ingredients for a framework of certified quantum computation. This is
inspired by the Haskell QIO monad but adds the necessary structure to
reason about quantum algorithms expressed in this monadic form. It
also addresses the issue of being able to check the side-conditions to
achieve unitarity statically.

This development can be viewed as a proof of concept which needs
further refinement to make it usable in practice. Our aim is to
provide a scalable, certifiable and efficient implementation of the QIO
framework in a dependently typed system, such as Agda, Coq or Epigram
2.  One of the first issues to be addressed is the choice of system,
each of the above has specific advantages and disadvantages: Coq is
the most stable system with the most extensive library, Agda is
especially good in supporting programming, while Epigram 2 is going to
offer a novel extensional and generic Type Theory but is currently at
a very experimental stage.

One of the issues we need to address is using an implementation of
constructive reals replacing the unsound use of floating point numbers
and postulates in Green's PhD prototype. Here, Coq would provide clear
advantages since we could build on the implementations from the
constructive mathematics library \citemain{CORN}, such as the constructive
reals \citemain{realsincoq}. 
Recently Li Nuo has implemented a
library for constructive reals and complex numbers for Agda
but there isn't much experience with using it.

Another issue we need to address is the fact that we need to extend
the QIO framework with a notion of distributed quantum computation to be
able to model quantum protocols adequately. We already have experience
on how to model conventional distributed programming in a dependently
typed setting \citemain{alti:tfp08} and we are hopeful that it should be possible to
transfer this approach to a quantum setting.

The formalisation of quantum mechanics was one of the motivating
examples in our recent paper on relative monads \citetrack{alti:fossacs10}.
We envisage that this pattern will be useful in our implementation.

\subsubsection*{Research challenges}
\label{sec:rsearch-challenges}

The main research challenge is to develop a framework which is useful for
researchers in the field of quantum computing who are not specialists
in Type Theory. The integration of concurrency and quantum information
processing in a type theoretic framework is also an important issue 
we hope to address.

\subsection*{WP2 : Development of certified quantum algorithms and protocols }
\label{sec:wp2}

The development of the QIO framework (WP1) needs to be informed by
experiments in using the framework. 
We will start by using the QIO framework for the
certified development of classical quantum algorithms and protocols
such as teleportation, super dense coding, Deutsch's algorithm,
Simon's algorithm, Shor's algorithm and Grover's algorithm. While the
algorithms are well known, to our knowledge nobody has yet formally
verified them using an interactive theorem prover. We also aim for an
effective way to develop certified quantum algorithms, based on our
experience in using a type theoretic framework for conventional
algorithms. I.e. instead of first writing an algorithm and then
verifying it, we aim at an integrated process where algorithm and
verification are developed in tandem.

Having honed our skills on well established quantum algorithms we
intend to use our framework to verify a more recent algorithm or
protocol whose correctness may not yet be accepted by the community ,
e.g.  examples can be found in \citemain{cacm} ---
thus proving the usefulness of our approach for cutting edge research
in quantum algorithms. This will also provide a test-case for
dependently typed programming which has shown its usefulness in a
conventional setting \citemain{gonthier:four-colour-paper}.

In traditional protocol verification, formal methods have proved to be
of great help in the design and  consolidation of protocols. This has
not yet been brought over to work with quantum protocols. Our project
would be a further step in this direction, which is extremely timely
given the emergence of quantum protocols and their potential
usefulness in the near future.

Simon Gay and his group in Glasgow are exploring quantum process
calculi. We would like to collaborate with them and investigate how
aspects of their calculus and their case studies can be expressed in
our framework.

% %%{\color{red}
% Recent motivation in finding new algorithms
% comes from the fact that it has been shown many
% quantum algorithms that are able to offer an exponential speed up over
% the best known classical counter-parts are all specific cases of the
% more general \emph{hidden subgroup problem}. E.g. Shor's algorithm
% finds the symmetries of the group formed using addition modulo $N$.
% This discovery has lead to new algorithms being found (see
% \citemain{cacm}) which gives us a plenty of choice for algorithms to
% verify in the QIO framework.
% %%}

\subsubsection*{Research challenges}

The main challenge is to demonstrate that a type theoretic framework
is capable of efficiently certifying standard and novel quantum
algorithms and protocols. 

\subsection*{WP3: Certified compilation (PhD project)}
\label{sec:wp3}

The QIO framework as we have sketched it so far, provides means to
simulate and reason about quantum computations. However, instead of
introducing yet another formalism for quantum computation we plan to
implement translations, i.e. compilers into different existing
formalisms such as the circuit model, Measurement based, or one-way
quantum computation \citemain{oneway,meascalc},
adiabatic quantum computation \citemain{adiabatic} and topological quantum
computation \citemain{topological}.

Exploiting the expressiveness of our dependently typed implementation
language we will be able to implement \textbf{certifying} compilers.
A certifying compiler does not just produce object code but it also
produces certificates that the object code behaves in the same way as
the source code (e.g a rather large example is CompCert
\citemain{compcert-back}). In particular such a compiler can translate verified
properties of the source code into verified properties of the object
code. 

Having certifying compilers with different target architectures
provides a setup in which we can study the different approaches in one
framework. We can compare their relative efficiency or exploit
translation to optimise quantum code, e.g. by applying optimisations
which have been discovered in measurement based quantum computation to
other approaches \citemain{meascalc,pqc}.

We plan to collaborate with our colleagues who are working on
measurement based quantum computing: Kashefi in Edinburgh and
Brown at Imperial college who have the theoretical background and
would benefit from the additional guarantees and precision provided
by a formally verified translation in the QIO framework.

The development of a certified compiler for at least one of the
potential target structures would provide an ideal subject for a PhD
thesis. 

%  We plan to apply this methodology to quantum algorithms,
% targeting first quantum circuits and measurement based quantum
% computing. The QIO framework plays two roles here: first of all we can
% formalise the compilation process within our framework which will
% force us to spell out some of the basic but implicit assumptions of
% Quantum Information Processing in detail. Second, based on this we will be able to
% enhance our framework by adding certifying compilers targeting
% different low level formats. Based on feedback with our colleagues we
% would like to enhance this by taking recent work on realistic physical
% realisations of Quantum Information Processing into account.
% %%{\color{red} 

% Researchers working on realising an actual physical implementation of
% a quantum computer are working with many different ideas. It is not
% just the implementations that differ, but also the underlying models of
% quantum computation. The standard circuit model of quantum
% computation, and Measurement based, or one-way quantum computation
% \citemain{oneway,meascalc}, are just two of a few models that we would
% ultimately like to look at. Other models include adiabatic quantum
% computation \citemain{adiabatic} and topological quantum computation
% \citemain{topological}. 
% %%}

\subsubsection*{Research challenges}

The main research challenge here is to represent different realisations
of quantum computing in one system to be able to relate and compare
them formally. Another challenge is to formally develop methods
for compiling quantum programs into an efficient representation.

\subsection*{WP4 : High level structures}
\label{sec:wp4}

The effective development of quantum algorithms and protocols relies
on the effective use of high level patterns reflecting an abstract
view of quantum physics. We have in the past proposed the language QML
which addresses the issue of control of decoherence in quantum
algorithms. As a first step we plan to provide a certified
implementation of QML within the QIO framework. We have also realised
that another central component of quantum algorithms and protocols is
the change of base, which in functional programming terms corresponds
to different, so called \emph{views} of a data structure. Indeed, in a
conventional setting views were one of the motivating programming
patterns for dependently typed programming \citemain{vftl}. We plan to
explore the use of quantum views for the high level development of
certified quantum algorithms - clearly this aspect is closely linked
with WP2. 

The group at Oxford lead by Abramsky and Coecke has made substantial
progress in not only providing a high level account of quantum
structures using a categorical perspective, namely dagger compact
closed categories, but they have also exploited this approach to
develop a graphical calculus which supports and simplifies reasoning
about quantum systems. We envisage that they could benefit from using
our framework to formally develop and certify their approach.

Dixon and Duncan are developing a graphical reasoning tool
(Quantomatic \citemain{quanto}) which is based on the Oxford approach. 
The interaction
of these two projects would provide an excellent synergy: Dixon and
Duncan are developing powerful, interactive reasoning tools which could
be embedded in our expressive framework. At the same time their
approach would provide a means to automate developments in the QIO
framework. 

\subsubsection*{Research challenges}

The main challenge is to represent high level approaches to quantum
programming within a type theoretic framework. Another challenge is to
exploit automated and graphical reasoning tools modelling quantum
systems to generate proofs for an interactive theorem prover. 

\subsection*{WP5: Workshop}
\label{sec:wp5:-summerschool}

Halfway through the project we plan to organize a workshop on
\emph{Formal Methods in Quantum Computing}. This will be an
opportunity to bring together researchers from the interactive theorem
proving community and from the quantum computing area. Naturally we
plan also to present preliminary results from our own work and invite
our collaborators.

% Our interaction during the EPSRC supported QNET framework provided us
% with valuable contacts to other groups. We would like to maintain and
% also intensify these contacts during the lifetime of the proposed
% project. There are a number of interesting interactions: the team of
% Abramsky and Coecke at Oxford has made substantial

% We do not intend to ignore the work of our colleagues at Oxford who
% have made substantial progress in not only providing a high level
% account of quantum structures using a categorical perspective, namely
% dagger compact closed categories, but who have also exploited this
% approach to develop a graphical calculus which supports and simplifies
% reasoning about quantum systems. We believe that they could benefit
% from using our framework to formally develop and certify their
% approach. We are looking forward to a a fruitful interaction in this
% area.

% Our interaction during the EPSRC supported QNET framework provided us
% with valuable contacts to other groups, in particular we would
% like to explore links to 
% %%{\color{red}
% the graphical calculus by Abramsky and Coecke in Oxford, the
% measurement calculus by Kashefi (et al.) in Edinburgh, and the work on
% verifying quantum protocols by Gay, Nagarajan, and Papanikolaou in
% Glasgow and Warwick \citemain{vqp}.
% %%}

% \subsection*{WP1: High-level structures in QIO}
% Many quantum algorithms depend on some sort of \emph{change of basis}
% that is used to extract the required information from a larger
% super-position of states. One such example, is the use of the quantum
% Fourier transform in Shor's period-finding subroutine. These changes of
% basis equate nicely to the \emph{views} that we have in dependently
% typed languages such as Agda.  

% \subsection*{WP2: Modelling the field of complex numbers in a
% dependently-typed setting}
% One of the main draw backs of the current implementation of QIO is its
% dependency on an underlying implementation of floating point
% numbers. As such, it is impossible to prove certain properties of the
% complex numbers that are necessary for giving us a sound bedding on
% which to build a formally verified language. Although we can currently
% postulate these facts, namely the properties that define the complex
% numbers as an algebraically closed field, it leads to a very
% un-modular way in which to reason about the QIO programs produced. An
% important goal of this project will be to study in more depth a way of
% defining the field of complex numbers in a dependently typed setting
% that gives us easy access to the properties we wish to prove.

% \subsection*{WP3: Compiling QIO}
% Having QIO embedded in a parent language has been very useful for
% exploring what can be done. However, one of our goals is to have QIO
% as a fully independent language. This will help to separate the
% denotational syntax of QIO programs from any underlying operational
% semantics, as well as enable QIO programs to be compiled to any
% underlying model of quantum computation. For example:

% \subsubsection*{Measurement patterns}
% One way quantum computation \citemain{oneway} differs from the standard
% model of quantum computation by making use of a large entangled state
% or \emph{graph state} as a resource. Computations are achieved by
% measuring the qubits in a graph state in a certain order, whereby the
% basis in which a measurement is made can depend on previous
% measurements. The Measurement calculus \citemain{meascalc} has been
% proposed as a way of defining the \emph{measurement patterns} that are
% necessary to propagate a specific computation. We plan to have these
% measurement patterns as one of the possible semantics for a compiled
% QIO computation.

% \subsubsection*{Quantum circuits}
% By far the most common model of quantum computation in use currently,
% is that of quantum circuits. We plan to define a universal set of typed
% circuits that can be used as one of the possible semantics for a
% compiled QIO computation.

% \subsection*{Simulation}
% As well as being able to compile QIO programs it is nice to be able to
% test running them. For this purpose, we would have an operational
% semantics that equates to the simulation of a quantum computation on a
% classical system. One advantage of this process is being able to
% simulate a probability distribution of all possible results from
% running the computation.

% \subsection*{WP4: Formally verified quantum algorithms}
% Having QIO as a fully fledged formally verified quantum programming
% language will enable us to develop quantum algorithms using it. As
% well as the underlying structure of QIO programs being formally
% verified, we would like to use the dependent type system to
% verify the algorithms written using QIO. For example, an
% implementation of Deutsch's algorithm \citemain{1992-deutsch} could be 1. To demonstrate that a type theoretic framework is capable of
efficiently certifying standard and novel quantum algorithms and
protocols.
2.  To develop a framework for certified quantum information
processing which is useful for researchers in the field of quantum
computing who are not specialists in Type Theory.  
3. The integration of concurrency and quantum information
processing in a type theoretic framework.
4. Represent different realisations of quantum computing in one system
to be able to relate and compare them formally.
5. To formally develop methods for compiling quantum programs into an
efficient representation.
6. To represent high level approaches to quantum programming within a
type theoretic framework.
7. To exploit automated and graphical reasoning tools modelling
quantum systems to generate proofs for an interactive theorem prover.
% verified to give a solution to the problem of whether the input
% function is constant or balanced. This form of verified code is likely
% to be a major asset, as quantum machines, as a resource, will be...

% \subsection*{WP5: Types for quantum computation}
% In order to develop QIO as an independent programming language, it is
% necessary to look at both the classical types and quantum types that
% it will use. Quantum data types can, in general, be defined in terms of
% operations on classical structures over quantum bits, or qubits. For
% example, a pair of qubits is sufficient to model the Bell states, or a
% list of qubits can be used to represent a form of quantum integer. It
% is therefore an important question to ask which classical structures
% and types we would like in QIO. This question also includes the
% dependent types that are required in order to give our unitary
% operations a formal verification.

% \subsection*{WP6: Quantum protocols and QIO}
% \subsubsection*{Quantum Error Correction}
% The Quantum IO Monad gives us an idealised model of quantum
% computation, where decoherence can only occur explicitly in
% measurement operations. It is likely that any \emph{real} quantum
% system will have to deal with decoherence from other sources, such as
% interaction with the environment. As such, there is much work on
% quantum error correction methods, for dealing with unwanted
% decoherence within a quantum system. We would like to be able to use
% QIO to look at these protocols, including a version of QIO that is
% able to model these other forms of unwanted decoherence.

% \subsubsection*{Quantum Cryptography}
% Quantum crytpography is one of the fastest growing areas of study in
% quantum computation. There are already provably secure quantum
% protocols such as BB84 \citemain{BEN84}, which are in use
% commercially over fibre-optic networks. We would like to use QIO to
% model and study these protocols.

% \subsection*{PhD project}
% The PhD student will primarily work on work packet 6.

\section{Relevance to Beneficiaries}

Our primary beneficiaries are other researchers working in quantum
computing. The QIO framework we are developing will provide a tool
they can use to formally develop and certify their contributions in
different areas: the development of new quantum algorithms, the
development of physically realistic approaches or the development of
high level approaches to Quantum Information Processing. We also
envisage useful interactions with physicists, exploiting our existing
contacts to the quantum groups at Nottingham and Imperial college.

We have already exploited the Haskell QIO monad to teach quantum
computing at Nottingham. The QIO framework we would be developing in
this project would provide further opportunities to educate students
and the interested public in quantum information processing. This is
especially important, since prototypes are limited and hard to
construct. 

It is generally believed that the development of a universal quantum
computer is at least five, if not ten, years away. In this respect the
benefits of our project are rather long term. On the other hand the
use of quantum based systems for unconditionally safe key
distribution, a corner stone for safe communication protocols, is
already the subject of startup companies \citemain{qci}. 
In this area our framework could provide commercially
relevant certificates within the lifetime of our project.

\section{Dissemination and Exploitation}

The field of quantum computation is a highly interdisciplinary
subject, with much work in both Computer Science, and Physics
departments. As we have already indicated our work relies very much on
the interaction and the input from our colleagues. Hence we plan to
integrate extended exchanges with 
%%{\color{red}
some of the groups that were also
members of QNET, e.g. Oxford, Edinburgh, Glasgow and Warwick,
%%}
to transfer information both
ways. 

Additionally we will use the traditional way of disseminating
research results, with workshop and conference papers and journal
publication. There are many workshops and conferences to which we
could submit our research, some specific to quantum computation
include QPL, QICS workshops, the Reversible
Computing workshop, QUOXIC meetings, Categories, Logic and Foundations
of Physics workshops. We plan to publish draft papers on arXiv.org
and publish journal papers in computer science journals such as
TCS, JFP etc. 

We have already successfully used the Haskell QIO monad to teach
quantum computing to undergraduates. We plan to develop this further,
providing course material based on our framework both on the
undergraduate and graduate level (for the latter we plan to offer a
course as part of the Midland Graduate School).

The workshop on \emph{Formal Methods in Quantum Computing} will 
concentrate on topics related to the formal development of quantum
software, and will also be used to disseminate our results. 

Since the QIO framework is a central component of the project, we plan
to make the software available on an open source licence.  We also
plan to support and interact with the interested public via wikis and
shared software repositories.

{\small 
\bibliographystylemain{abbrv}
\bibliographymain{proposal} 
}

% {\small
% \bibliographystyle{abbrv}
% \bibliography{proposal} 
% }
\end{document}
