\documentclass[a4paper]{article}

% upto two sides of A4

\usepackage{eepic}
\usepackage{times} \usepackage{mathptm}  
\usepackage{amssymb}
\usepackage{latexsym}
\usepackage{url}
\RequirePackage{vmargin}
\setmarginsrb{1.9cm}{2cm}{1.9cm}{2cm}{0pt}{0pt}{0pt}{0pt} %jg 2cm all round
  %ctm shaves another mil left and right
\setlength{\columnsep}{1cm}

%jg redefined title layout
\makeatletter
\def\maketitle{%
\begin{flushleft}
\LARGE\bfseries \@title \bigskip \\
\Large\mdseries \@author \medskip
\end{flushleft}%
}
\makeatother

\setlength{\marginparwidth}{1cm}
\setlength{\marginparsep}{3mm}
\def\marginnote#1{\marginpar{\scriptsize\itshape\raggedright#1}}

%%%\renewcommand{\baselinestretch}{0.9641}\small\normalsize
% \renewcommand{\baselinestretch}{0.972373}\small\normalsize %jg ick!
\makeatletter
%%%\setlength{\parskip}{0\p@ \@plus 0.98\p@}
% \setlength{\parskip}{0.75\parskip} %jg prefer a parindent
\renewcommand{\section}{\@startsection
 {section}%
 {1}%
 {0mm}%
 {-\baselineskip}%
 {0.5\baselineskip}%
 {\Large\bfseries}%
 }%
\renewcommand{\subsection}{\@startsection
 {subsection}%
 {2}%
 {0mm}%
 {-0.75\baselineskip}%
 {0.375\baselineskip}%0.01
 {\large\bfseries}%
 }%
\makeatother

\pagestyle{empty}

% \parskip 0.05in \parindent 0in %jg prefer a parindent
\newcommand{\sparadrap}[1]{\par\mbox{\textbf{#1}$\;\;$}}
\newcommand{\squashit}{\itemsep=0pt
  \topsep=0pt \parskip=0pt \parsep=0pt}
\let\ITEMIZE\itemize \def\itemize{\ITEMIZE\squashit}

\newcommand{\remph}{\emph}
\newcommand{\demph}{\textbf}

\newcommand{\bsize}[2]{{#1\textbf{#2}\par}}
\newcommand{\bL}{\vspace{0.1in}\noindent\bsize{\Large}}
\newcommand{\bl}{\bsize{\large}}
\newenvironment{titemize}{
  \begin{list}{\ensuremath{\bullet}}{
  \setlength{\topsep}{1pt}
  \setlength{\parsep}{0pt}
  %\setlength{\parskip}{0pt}
  \setlength{\itemsep}{1pt}
  \setlength{\partopsep}{0pt}}
}
{
  \end{list}
}

\usepackage{lscape}
\usepackage{pstricks}
\usepackage{pst-grad}
%%\usepackage{pst-xkey}
\usepackage{multido}

\makeatletter
\makeatother

\newcommand{\surl}[1]{\small{\url{#1}}}
\newcommand{\parametris}{parametriz}




\title{Quantum Types for Quantum Programs:\\
 \Large Towards a Framework for Certified Quantum Information Processing
}
\author{(Justification of Resources) --- Thorsten Altenkirch and Alexander S. Green}
\date{}

\begin{document}

\twocolumn[\maketitle]{

\section*{Personnel}
\label{sec:personnel}

The research suggested in the proposal is based on the PhD of
Alexander S. Green supervised by Thorsten Altenkirch. 
Altenkirch as the lead investigator will spend 15\% of his time on the
project, on which Green would be a full time RA.

Since the project involves both theoretical work and the
implementation of a software package, a running time of 4 years is
justified.

\subsection*{Conference attendance}
\label{sec:travel}

We plan to attend conferences and workshops relevant to our research,
with the aim to learn about progress in the area and to dissiminate
our own results. Topical workshops are often most productive, but it
is often difficult to predict exactly when and how often they will
take place. We would attend meetings on quantum computing, and in
particular meetings on the interaction of quantum information
processing and logic (e.g. QPL (Quantum Physics and Logic), QUOXIC
(Quantum meetings of Imperial College and Oxford); the
Reversible Computation workshop, the workshops on Categories, Logic
and Foundations of Physics), meetings related to dependently typed
programming (e.g. TYPES workshops, DTP (Workshop on Dependently Typed
Programming; AIM (the Agda Implementers Meeting); the Coq workshop);
meetings on functional programming (e.g. TFP (Trends in Functional
Programming) ; ICFP (the International Conference on Functional
Programming)) and general Computer Science conferences, especially if
they include a session on quantum computing (e.g. ( CiE (Computability
in Europe), POPL (Principles of Programming Languages); LICS (Logic in
Computer Science)).

We estimate that the PI and the RA will attend 1 international and 1-2
UK/EU workshops or conferences each year. We estimate the costs for
attending an international conference as \pounds 1800, and national/EU
conferences as \pounds 1000. Thus, over the running time of the
project we would have 8 international and 12 national/EU conference visits.

\subsection*{Collaboration}
\label{sec:collaboration}

We plan to collaborate closely with Ross Duncan in Strathclyde,
and with Samson Abramsky and Bob Coecke in Oxford. We also plan to
collaborate with Peter Selinger at Dalhousie. This
collaboration is important to obtain early feedback on the usefulness
of our design (WP1) and specifically on using high level structures
(WP4), i.e. Coecke's and Abramsky's categorical models and their
implementation in Dixon's and Duncan's \emph{Quantomatics}.  

For UK based collaboration, we envisage 4 visits lasting approximately
one week, and budget \pounds 800 each for travel (appr. \pounds 150) ,
accomodation (appr. \pounds 400) and subsistence (appr. \pounds
250). i.e. \pounds 3200 over the lifetime of the project.

For collaboration with Selinger in Halfiax, Nova Scotia, we
envisage 1 visit lasting approximately two weeks, and budget \pounds
2150 each for travel (appr. \pounds 850) , accomodation (appr. \pounds
800) and subsistence (appr. \pounds 500). i.e. \pounds 2150 over the
lifetime of the project.

% \subsection*{Workshop and Spring School}
% \label{sec:worksh-summ-scho}

% We will organize a workshop on \textit{Formal Methods in Quantum Computing} 
% in Nottingham during the middle of the project to discuss
% our work with other researchers but also to obtain valuable input on
% our work. We would like to fund travel
% and accommodation (est. $\pounds 1000$ each) for two invited speakers.

% We also plan to present a course at the Midland Graduate School and
% include \pounds 800 to cover accomodation and travel for this.

\section*{Equipment and consumables}
\label{sec:equipment}

The research proposed here  necessitates the development of experimental
prototype software, seldom optimized at first, and requiring
frequent time-consuming recompilation. Correspondingly, we
request funding for high spec desktop PCs and laptops
to support these investigations effectively.

We will equip each of the participants with a laptop computer
(est. cost $\pounds 1500$) for presentations and work while not at the
office.  Laptops are essential for joint work at the miniworkshops and
during visits and are not supplied by the university.
The researcher will also require a high spec desktop computer (at
$\pounds 1000$).

\end{document}

